\chapter{Conclusion} % Main chapter title
%\addcontentsline{toc}{chapter}{Conclusion générale}  
%\startcontents[chapters]
%\pagestyle{plain} % remove headers/footers from the chapt\markboth{Left}{Right}
 


%\pagestyle{fancy}
%\renewcommand{\chaptermark}[1]{\markboth{Chapter \thechapter. #1}{}}
%\renewcommand{\sectionmark}[1]{\markright{\thesection\ #1}}


Dans ce rapport, nous avons exploré en profondeur le marché des taux d'intérêt, un domaine crucial de la finance qui influence de manière significative de nombreux aspects de l'économie mondiale. En commençant par une définition claire du marché des taux d'intérêt,  nous avons établi les bases nécessaires à la compréhension de ce domaine complexe.

En classant les taux d'intérêt en différentes catégories sur la base de leurs caractéristiques, nous avons offert un aperçu approfondi des différentes dimensions de ce marché dynamique. Cette analyse nous a permis de mieux saisir l'ampleur de l'influence des taux d'intérêt sur les décisions d'investissement et de prêt.

En détaillant les modèles stochastiques de type "short" et "forward", notamment les modèles affine, Hull-White et binomial forward, nous avons fourni des outils puissants pour la modélisation et l'analyse des taux d'intérêt. Ces modèles nous permettent d'identifier et de comprendre les principales forces qui façonnent l'évolution des taux d'intérêt, ainsi que les implications économiques qui en découlent.

En conclusion, le marché des taux d'intérêt est un domaine complexe mais essentiel de la finance, qui influence de manière significative l'ensemble de l'économie mondiale. En comprenant les mécanismes et les tendances qui le sous-tendent, les acteurs du marché peuvent prendre des décisions éclairées et contribuer à la stabilité et à la croissance économique à long terme.
