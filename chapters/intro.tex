\chapter*{Introduction} % Main chapter title
\addcontentsline{toc}{chapter}{Introduction}  

À l'aide du livre «Financial mathematics : theory and problems for multi-period models»\cite{pascucci_financial_2012}, on va étudier le taux d'intérêt et ses modèles.

Les taux d'intérêt et la structure des échéances jouent un rôle central dans le fonctionnement des marchés financiers et dans les décisions de politique monétaire. Comprendre leur comportement et les modéliser de manière précise est crucial pour les investisseurs, les gestionnaires de risques, les décideurs politiques et les praticiens du domaine financier. Ce rapport se propose d'explorer en profondeur ces concepts fondamentaux, en mettant l'accent sur les méthodes de modélisation qui permettent de saisir la complexité et la dynamique des marchés des taux d'intérêt.

La modélisation des taux d'intérêt est une discipline interdisciplinaire qui fait appel à la théorie financière, à l'économétrie, aux mathématiques financières et à la statistique. Elle vise à expliquer et à prédire les variations des taux d'intérêt à différentes échéances, ainsi qu'à comprendre les relations entre ces taux et d'autres variables économiques et financières.

Au cours des dernières décennies, un large éventail de modèles a été développé pour capturer les caractéristiques complexes des taux d'intérêt et des structures d'échéances. Ces modèles vont des approches traditionnelles telles que le modèle de taux d'intérêt à terme à des modèles plus sophistiqués basés sur la théorie des probabilités, tels que les modèles à facteurs ou les modèles à sauts.

Dans ce rapport, nous commencerons par la définition de marché des taux d'intérêt. Nous présenterons les taux d'intérêt en introduisant les T-bonds et nous classerons les taux d'intérêt en différentes catégories sur la base de différentes propriétés et caractéristiques.

Ensuite, nous détaillerons deux types de modèles stochastiques d'évolution des taux d'intérêt, à savoir les modèles de type "short" et "forward". Pour le modèle short, nous intéressons au modèle affine et au modèle Hull-White. Pour le modèle forward, nous intéressons au modèle binomial forward. La consruction des ces modèles est pour le but d'identifier les différentes grandeurs économiques associées aux taux d'intérêt.